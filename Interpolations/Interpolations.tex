\documentclass{article}
\usepackage{amsmath}
\usepackage{amssymb}
\title{Interpolations\\in\\Geometric Algebra}
\author{Spencer Parkin}
\addtolength{\oddsidemargin}{-.575in}
\addtolength{\evensidemargin}{-.575in}
\addtolength{\textwidth}{1.0in}
\addtolength{\topmargin}{-.575in}
\addtolength{\textheight}{1.25in}
\begin{document}
\maketitle

\newcommand{\V}{\mathbb{V}}

\section*{Linear Interpolations of Vectors}

Given $a,b\in\V$, we seek a reasonable way to
interpolate between them.  By this we mean
that if we're given a real value $0\leq t\leq 1$,
then we want to find a vector having the percentage
$100(1-t)$ of the vector $a$'s charactristics, and
the percentage $100t$ of the vector $b$'s characteristics.
Born from this idea is the linear interpolation, given
as simply a weighted average of the two vectors.
\begin{equation*}
\mbox{lerp}(a,b,t) = a(1-t)+bt = a+(b-a)t
\end{equation*}
From this it is easy to see that an interpolated vector
points along the line between the vectors $a$ and $b$.
While for some appliations this may be desired, for others
this presents a few problems.  To begin with, we may find
it unacceptable that the rate of change in angle between
the interpolated vector and one of the vectors $a$ or $b$
is non-linear with respect to a change in $t$.  We may also
find it undesirable for the length of the interpolated
vector to very with $t$.
These attributes of the linear interpolation motivate us to
find the spherical linear interpolation.

\section*{Spherical Linear Interpolations of Vectors}

Given a pair of linearly independent vectors $a,b\in\V$, each
of unit length, we wish to uniformly interpolate between them along
the shortest arc of the unit sphere connecting
the tips of $a$ and $b$ by the normalized index $0\leq t\leq 1$.  Notice that
such an arc exists by our requirement that $a$ and $b$ are not parallel to
one another.  Using what we know about rotors, it is easy to
construct the desired interpolation.  We begin by constructing
the rotor that will rotation $a$ toward $b$ so that the ratio
of the angle made between the interpolated vector and $a$ with
the angle made between $a$ and $b$ is exactly $t$.  The desired
rotor is given by
\begin{equation*}
R(a,b,t) = \exp\left(-\frac{a\wedge b}{|a\wedge b|}(t\theta/2)\right),
\end{equation*}
where $\theta$ is the angle made between $a$ and $b$.
Here we have chosen the proper plane of rotation using the
blade $a\wedge b$, and then rescaled it to be of a magnitude
equal to half the desired angle of rotation.  The sign of
the blade is chosen so that the equation
\begin{equation*}
\mbox{slerp}(a,b,t) = R(a,b,t)a\tilde{R}(a,b,t)
\end{equation*}
rotates $a$ in the plane of $a\wedge b$ toward $b$ as required.
This is also a formula for the desired interpolation.
Though this formula is ready for use, it is worth seeing if
we can simplify the expression.  We can begin by leveraging
the fact that $a\wedge a\wedge b=0$.
\begin{align*}
\mbox{slerp}(a,b,t) &=
\exp\left(-\frac{a\wedge b}{|a\wedge b|}(t\theta/2)\right)a
\exp\left(\frac{a\wedge b}{|a\wedge b|}(t\theta/2)\right) \\
 &= \left(\cos(t\theta/2)-\frac{a\wedge b}{|a\wedge b|}\sin(t\theta/2)\right)a
\left(\cos(t\theta/2)+\frac{a\wedge b}{|a\wedge b|}\sin(t\theta/2)\right) \\
 &= a\cos^2(t\theta/2)-\frac{a\wedge b}{|a\wedge b|}a\frac{a\wedge b}{|a\wedge b|}\sin^2(t\theta/2) \\
 &+ \left(a\frac{a\wedge b}{|a\wedge b|}-\frac{a\wedge b}{|a\wedge b|}a\right)\cos(t\theta/2)\sin(t\theta/2) \\
 &= a(\cos^2(t\theta/2)-\sin^2(t\theta/2))
 + 2a\frac{a\wedge b}{|a\wedge b|}\cos(t\theta/2)\sin(t\theta/2) \\
 &= a\cos(t\theta)+a\frac{a\wedge b}{|a\wedge b|}\sin(t\theta) \\
 &= a\exp\left(\frac{a\wedge b}{|a\wedge b|}t\theta\right)
\end{align*}
Here we see that the double-sided transformation law for rotations
reduces to a single-sided transformation law when the vector to
be rotated is in the plane of rotation.  From this it is also clear
that we have been able to express the interpolated vector as
a linear combination of $a$ and $a\cdot(a\wedge b)/|a\wedge b|$,
a vector perpedicular to $a$.  The linear combination of
these vectors consists of $\cos(t\theta)$ and $\sin(t\theta)$,
making our result geometrically intuitive.  But since this formula requires
we find one of the basis vectors, it is worth looking for a form
of our result that expresses the spherical linear interpolation
as a linear combination of the vectors $a$ and $b$, which we have
to begin with.
To this end, we come up with the following matrix equation
\begin{equation*}
\left[\begin{array}{cc}\cos(t\theta)&\sin(t\theta)\end{array}\right]
\left[\begin{array}{c}a\\a\frac{a\wedge b}{|a\wedge b|}\end{array}\right]
=\left[\begin{array}{cc}f(t)&g(t)\end{array}\right]
\left[\begin{array}{c}a\\b\end{array}\right],
\end{equation*}
where the elements are vectors and matrix multiplication
is as usual, but using the geometric product instead of scalar multiplication.
Before we can proceed, we need to put the row-oriented matrices on a common basis.
It is convenient to use the components of each row-vector
parallel and perpendicular to $a$.  Our matrix equation then becomes
\begin{equation*}
\left[\begin{array}{cc}\cos(t\theta)&\sin(t\theta)\end{array}\right]
\left[\begin{array}{cc}a&0\\0&a\frac{a\wedge b}{|a\wedge b|}\end{array}\right]
=\left[\begin{array}{cc}f(t)&g(t)\end{array}\right]
\left[\begin{array}{cc}a&0\\(a\cdot b)a&-(a\wedge b)a\end{array}\right].
\end{equation*}
To see this, recall how the geometric product can be used to find parallel
and perpendicular vector components.
\begin{equation*}
b = ba^2 = (a\cdot b-a\wedge b)a = (a\cdot b)a - (a\wedge b)a
\end{equation*}
We now solve for the inverse of the row-oriented matrix on the right-hand side.
\begin{equation*}
\left[\begin{array}{cc}a&0\\(a\cdot b)a&-(a\wedge b)a\end{array}\right]
\left[\begin{array}{cc}w&x\\y&z\end{array}\right] =
\left[\begin{array}{cc}1&0\\0&1\end{array}\right]
\end{equation*}
Clearly, $w=a$ and $x=0$.
Solving for $y$, we get
\begin{align*}
 & (a\cdot b)a^2-(a\wedge b)ay = 0 \\
\implies& (a\wedge b)ay = a\cdot b \\
\implies& (a\wedge b)^2y = a(a\wedge b)(a\cdot b) \\
\implies& y = \frac{a(a\wedge b)(a\cdot b)}{(a\wedge b)^2}.
\end{align*}
Solving for $z$, we get
\begin{align*}
 & -(a\wedge b)az = 1 \\
\implies& -(a\wedge b)^2z = a(a\wedge b) \\
\implies& z = \frac{a(a\wedge b)}{-(a\wedge b)^2}.
\end{align*}
Having the inverse of the matrix, we're now ready to solve our original matrix equation.
\begin{align*}
\left[\begin{array}{cc}f(t)&g(t)\end{array}\right]
 &= \left[\begin{array}{cc}\cos(t\theta)&\sin(t\theta)\end{array}\right]
\left[\begin{array}{cc}a&0\\0&a\frac{a\wedge b}{|a\wedge b|}\end{array}\right]
\left[\begin{array}{cc}a&0\\\frac{a(a\wedge b)(a\cdot b)}{(a\wedge b)^2}&
\frac{a(a\wedge b)}{-(a\wedge b)^2}\end{array}\right] \\
 &= \left[\begin{array}{cc}\cos(t\theta)&\sin(t\theta)\end{array}\right]
\left[\begin{array}{cc}1&0\\-\frac{a\cdot b}{|a\wedge b|}&\frac{1}{|a\wedge b|}\end{array}\right] \\
 &= \left[\begin{array}{cc}\cos(t\theta)-\frac{a\cdot b}{|a\wedge b|}\sin(t\theta)
 & \frac{1}{|a\wedge b|}\sin(t\theta)\end{array}\right]
\end{align*}
We can now write $g(t)=\sin(t\theta)/\sin(\theta)$, and we are not
far from a similar result for $f(t)$.
\begin{equation*}
f(t) = \frac{\sin(\theta)\cos(t\theta)-\cos(\theta)\sin(t\theta)}{\sin(\theta)} =
 \frac{\sin((1-t)\theta)}{\sin(\theta)}
\end{equation*}
Our formula for the spherical linear interpolation now becomes
\begin{equation*}
\mbox{slerp}(a,b,t) = \frac{a\sin((1-t)\theta)+b\sin(t\theta)}{\sin(\theta)},
\end{equation*}
which is a nice result!  It may be more efficient to use
$f(t)$ in the non-simplified form, since $\cos(\theta)$ and $\sin(\theta)$
can be computed simultaneously, as can $\cos(t\theta)$ and $\sin(t\theta)$.
That the interpolated vector is always of unit-length follows directly
from our construction of the interpolation formula, as do the other properties
of the interpolation.

Notice that our derivation did not depend on the dimensionality of
the vectors $a$ and $b$.  This means that our formula works for any
pair of unit-length, linearly independent vectors pointing to the
surface of an $n$-dimensional hyper-sphere!
What we should now investigate is the idea of interpolating between
blades in general of the same grade.

\section*{Spherical Linear Interpolations of Blades}

For simplicity, we will not invent an interpolation method that
blends between blades taken from disjoint sub-spaces.
Even more restrictive, for there to be a spherical linear
interpolation between two blades, we will require the existence
of a rotor that transforms one blade into the other.  This is
the basis for the idea behind the interpolation.  If there exists
such a rotation, then by performing a partial rotation we can
generate an interpolation between the two blades.  This is consistent
with the approach taken to derive the spherical linear interpolation
between vectors.  This may seem too restrictive at first, but all that
we're really requiring here is that the two blades in question have
the same hyper-volume, live in the same sub-space, and have the same grade.
Recall that the shape of a blade in its hyper-plane is open to interpretation.

Without further delay, let $A$ and $B$ be $n$-blades meeting the said criteria.
It follows that there exists a rotor $R$ such that $RA\tilde{R}=B$.
If we can then write
\begin{equation*}
A = \prod_{k=1}^n a_k,
\end{equation*}
where each of $a_1$ through $a_n$ is a vector making up a geometric product for $A$,
then we can write
\begin{equation*}
B = \tilde{R}\left(\prod_{k=1}^n a_k\right)R = \prod_{k=1}^n \tilde{R}a_k R.
\end{equation*}
What this shows is that we can derive our spherical linear interpolation of blades
in terms of our spherical linear interpolation of vectors.
Using the rotor-valued function $R(a,b,t)$ defined earlier, we have
\begin{align*}
\mbox{slerp}(A,B,t) &= \prod_{k=1}^n R(a_k,\tilde{R}a_kR,t)a_k\tilde{R}(a_k,\tilde{R}a_kR,t) \\
 &= \prod_{k=1}^n\mbox{slerp}(a_k,\tilde{R}a_kR,t) \\
 &= \frac{1}{\sin^n(\theta)}\prod_{k=1}^n(a_k\sin((1-t)\theta)+\tilde{R}a_kR\sin(t\theta)),
\end{align*}
where $|A||B|\cos(\theta)=A\cdot B$.  I'm not sure what else to say about this.
It's not a particularly interesting result, but it was worth thinking about.

% show that the interpolated blade has the same magnitude

\section*{Rotor Interpolation}

Traditional interpolation of rotors (or of quaternions) is done
by interpretating the scalar components as a linear combination
of a vector basis for a vector on the unit hyper-sphere, and then using the spherical
linear interpolation method developed earlier.  This has a clear geometric
meaning with respect to vectors, but what is it doing to the rotor?
The traditional method does not make this immediately clear.
Notice that the formula developed for interpolating between blades
in the previous section is inadequite to the task at hand.  This is
because we need to be able to interpolate rotor blades of different magnitude.
Another way to see why is to realize that when a rotor is used to rotate a rotor,
the rotated rotor still rotates things by the same angle.  Thus the idea of
incrementaly rotating one rotor into another will not work.
Before investigating the geometric interpretation of the traditional method, let
us try to come up with a reasonable interpolation method.
\begin{equation*}
\mbox{blend}(R_0,R_1,t) = \exp\left(\mbox{lerp}(|B_0|,|B_1|,t)\;
\mbox{slerp}\left(\frac{B_0}{|B_0|},\frac{B_1}{|B_1|},t\right)\right)
\end{equation*}
This formula lets us attach a clear geometric meaning to the interpolated rotor.

\end{document}