\documentclass[12pt]{article}

\usepackage{amsmath}
\usepackage{amssymb}
\usepackage{amsthm}
\usepackage{graphicx}
\usepackage{float}
\usepackage{algpseudocode}

\title{Sudoku Puzzle Theory}
\author{Spencer T. Parkin}

\numberwithin{equation}{section}

\newtheorem{theorem}{Theorem}[section]
\newtheorem{definition}{Definition}[section]
\newtheorem{corollary}{Corollary}[section]
\newtheorem{identity}{Identity}[section]
\newtheorem{lemma}{Lemma}[section]
\newtheorem{result}{Result}[section]

\begin{document}
\maketitle

\section{Introduction}

While finding solutions to Sudoku puzzles gives many solvers a sense of accomplishment, for some
such as myself, there is no real accomplishment in solving any one Sudoku puzzle,
no matter how hard it may be.  Instead, such a sense can only be attained by
producing a general theory of the puzzle in which a method for solving
any Sudoku puzzle naturally unfolds.  Having such a method, and being able
to implement it as a computer algorithm, one can put all Sudoku puzzles to rest.

Using the ideas of Peter Norvig in
his essay \cite{}, this paper is an attempt to develop such a theory, and
to apply it in the problem of solving the general Sudoku puzzle.

In the discussion to follow, we assume a knowledge of the general Sudoku puzzle,
and so procede with no introduction to it.

\section{Definitions and Lemmas}

A given instance of the Sudoku puzzle, in this paper, will be represented by a
valid set of decisions.
A decision $d$ is a decision to put a specific number into a specific
location of a Sudoku square.
A valid set of decisions $D$ has the property that for any two decisions
$d_0,d_1\in D$, the square associated with $d_0$ is not that of $d_1$,
and that if all decisions in $D$ were applied to an empty Sudoku square,
it would produce a valid, partially or fully completed Sudoku square.
Unless otherwise stated, all decisions sets in this paper are valid.

A valid set of decisions $D_0$ is said to be completable, if there exists
a valid set of decisions $D_1$ such that $D_0\cup D_1$ is a valid set of decisions
that, if applied to the empty Sudoku square, would product a
fully complete Sudoku square.  To be short, we say that the set of decisions
$D_0\cup D_1$ is complete.  If for a given completable set of decisions $D_0$, there
exists one and only one set of decisions $D_1$ such that $D_0\cup D_1$ is complete,
then we say that $D_0$ is uniquely completable.

Given any valid set of decisions $D$, we let $f(D)$ denote the set of all
decisions $d$ such that $D\cup\{d\}$ is a valid set
of decisions.  Notice that if $f(D)$ is the empty set, then $D$ is either complete
or not completable.  Note also, however, that non-completability of $D$ does not
imply the emptiness of $f(D)$.

For any given valid set of decisions $D$, if $d\in f(D)$ is a decision
such that $D\cup\{d\}$ is not completable, we call $d$ a bad decision in
the context of $D$, and a good decision in the context of $D$ otherwise.

\begin{lemma}\label{lma_d0_bad_imp_d1_bad}
Let $D_0$ and $D_1$ each be a completable set of decisions with $D_0\subseteq D_1$.
Then, if $d\in f(D_0)$ is a bad decision in the context of $D_0$ and $d\in f(D_1)$, $d$ is a bad decision
in the context of $D_1$.
\end{lemma}
\begin{proof}
Since $d$ is a bad decision in the context of $D_0$, we know that there does not exist a
set of decisions $D'$ such that $D_0\cup\{d\}\cup D'$ is complete.  Keeping this in mind, suppose,
to the contrary of the lemma, that there exists a set of decisions $D$ such that $D_1\cup\{d\}\cup D$
is complete.  But if we let $D'=(D_1-D_0)\cup D$, we then see that $D_0\cup\{d\}\cup D'$ is
complete, which contradicts the earlier establishment of the non-existance of such a set $D'$.
\end{proof}

\begin{lemma}\label{lma_d1_bad_imp_d0_bad}
Let $D_0$ and $D_1$ each be a uniquely completable set of decisions with $D_0\subseteq D_1$.
Then, if $d\in f(D_1)$ is a bad decision in the context of $D_1$, $d$ is a bad decision
in the context of $D_0$.
\end{lemma}
\begin{proof}
To begin, if $D_0\subseteq D_1$, then $f(D_1)\subseteq f(D_0)$ so that if $d\in f(D_1)$,
then $d\in f(D_0)$.  Then, since $d$ is a bad decision in the context of $D_1$, we know that
there does not exist a set of decisions $D'$ such that $D_1\cup\{d\}\cup D'$ is complete.
Keeping this in mind, suppose, to the contrary of the lemma, that there does exist a set of
decisions $D$ such that $D_0\cup\{d\}\cup D$ is complete.  But since $D_0$ is uniquely
complete, we must have $D_1\subseteq D_0\cup D$.  It follows that if we let $D'=D-D_1$,
then $D_1\cup\{d\}\cup D'$ is complete, which contradicts the earlier establishment of
the non-existance of such a set $D'$.
\end{proof}
Notice that the key to proving Lemma~\ref{lma_d1_bad_imp_d0_bad} was the uniqueness of the
completability of $D_0$.

\begin{lemma}\label{lma_bad_decision}
Let $D_0$ be a uniquely completable set of decisions, and let each of $D_1$ and $D_2$ be
a completable set of decisions with $D_0\subseteq D_1$ and $D_0\subseteq D_2$.
Then, if $d\in f(D_1)$ and $d\in f(D_2)$ and $d$ is a bad decision in the context of $D_1$, then $d$
is a bad decision in the context of $D_2$.
\end{lemma}
\begin{proof}
Clearly, each of $D_1$ and $D_2$ are uniquely completable, because $D_0$ is uniquely completable.
Then, by Lemma~\ref{lma_d1_bad_imp_d0_bad}, $d\in f(D_0)$ is a bad decision in the
context of $D_0$.  Lastly, by Lemma~\ref{lma_d0_bad_imp_d1_bad}, we see
that $d\in f(D_2)$ is a bad decision in the context of $D_2$.
\end{proof}

\section{Application}

Armed with Lemma~\ref{lma_bad_decision}, we proceed now to give a method
of solving any Sudoku puzzle.

A Sudoku puzzle may be thought of as an initial condition.  The solver is then
trying to find a completion of the square that obeys the constraint of this
initial condition.  In the language of this paper, such a condition is a
completable set of decisions $D_0$, and the solver is simply trying to find the
set $D_1$ such that $D_0\cup D_1$ is complete.  The algorithm we describe here
is a systematic search for this set.

We begin with the naive approach to solving the general Sudoku puzzle.
In the psuedo-code to follow, $D_1$ should be initially given as an empty set.

\begin{algorithmic}
\Function{Solve}{$D_0,D_1$}
	\If{$D$ is complete}
		\State \Return success
	\EndIf
	\ForAll{$d\in f(D_0)$}
		\State $D\gets D_0\cup\{d\}$
		\If{ success = \Call{Solve}{$D,D_1$} }
			\State $D_1\gets D_1\cup\{d\}$
			\State \Return success
		\EndIf
	\EndFor
	\State \Return failure
\EndFunction
\end{algorithmic}

While this algorithm may be correct, it is not practical due to the combinatorial
explosion found in trying to enumerate all possible search paths that this algorithm
would potentially need to go through before finally finding a solution to the puzzle.
What we'll find, however, is that Lemma~\ref{lma_bad_decision} offers a major optimization to the
code above, as realized by Peter Norvig in his essay \cite{}.  Applying this lemma, the revised
code becomes as follows, with the set $B$ being initially given as an empty set, and as
not necessarily being a valid set of decisions.

\begin{algorithmic}
\Function{Solve}{$D_0,D_1,B$}
	\If{$D$ is complete}
		\State \Return success
	\EndIf
	\ForAll{$d\in f(D_0)$}
		\If{ $d\not\in B$ }
			\State $D\gets D_0\cup\{d\}$
			\If{ success = \Call{Solve}{$D,D_1$} }
				\State $D_1\gets D_1\cup\{d\}$
				\State \Return success
			\Else
				\State $B\gets B\cup\{d\}$
			\EndIf
		\EndIf
	\EndFor
	\State \Return failure
\EndFunction
\end{algorithmic}

Explain why this works here and how it optimizes the problem...

\end{document}